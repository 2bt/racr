% Configure character output and input encoding:
\usepackage[T1]{fontenc}
\usepackage[utf8]{inputenc}

% Configure font:
\usepackage{lmodern}
\renewcommand*\familydefault{\sfdefault}
%\usepackage{txfonts}

% KOMA Script heading configuration:
\usepackage{scrpage2}
\pagestyle{scrheadings}
\automark[section]{chapter}

% Configure paragraph handling (no indentation):
\setlength{\parindent}{0pt}
\setlength{\parskip}{5pt plus 2pt}

% Configure table of contents depth:
\setcounter{tocdepth}{1}

% Provide support for non-eps pictures:
\usepackage{graphicx}

% Define colours (used for example as background colours in listings): 
\usepackage{color}
\definecolor{babyblue}{rgb}{0.93,.95,0.98}

% Configure listings:
\usepackage{listings}
\lstset{literate=
  {ö}{{\"o}}1
  {ä}{{\"a}}1
  {ü}{{\"u}}1
  {Ö}{{\"O}}1
  {Ä}{{\"A}}1
  {Ü}{{\"U}}1
}
\lstloadlanguages{Lisp}
\lstdefinestyle{generic-code}{
	basicstyle=\small,
	numbers=none,
	aboveskip=\bigskipamount,
	belowskip=\medskipamount,
	xleftmargin=7pt,
	%xrightmargin=7pt,
	tabsize=4,
	breaklines=false,
	columns=[c]flexible
}
\lstdefinestyle{lisp-style}{
	language=Lisp,
	basicstyle=\small\bfseries\ttfamily,
	commentstyle=\rmfamily\mdseries\fontshape{n},
	numbers=none,
	xleftmargin=7pt,
	xrightmargin=7pt,
	breaklines=false,
	showstringspaces=false,
	columns=[c]flexible
}
\lstdefinestyle{racr-api-function-heading-style}{
	language=Lisp,
	basicstyle=\small\bfseries\ttfamily,
	commentstyle=\rmfamily\mdseries\fontshape{n},
	aboveskip=\bigskipamount,
	belowskip=\smallskipamount,
	frame=lines,
	framesep=5pt,
	backgroundcolor=\color{babyblue},
	numbers=none,
	breaklines=false,
	columns=[c]flexible
}
\lstdefinestyle{lisp-appendix-style}{
	language=Lisp,
	basicstyle=\tiny\bfseries\ttfamily,
	commentstyle=\rmfamily\mdseries\fontshape{n},
	numbers=left,
	breaklines=true,
	showstringspaces=false,
	columns=[c]flexible
}

% Configure index:
\usepackage{makeidx}
\makeindex
\renewcommand\indexname{API Index}
\newcommand{\indexfunction}[1]{\index{#1@\lstinline[style=lisp-style];#1;}}

% Support for user specified line space (e.g., in the abstract):
\usepackage{setspace}